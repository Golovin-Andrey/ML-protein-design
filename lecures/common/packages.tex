%%% Поля и разметка страницы %%%
\usepackage{lscape}		% Для включения альбомных страниц
\usepackage{geometry}	% Для последующего задания полей
%\usepackage[a4paper, mag=1000,left=2.5cm, right=1cm, top=2cm, bottom=2cm, headsep=0.7cm, footskip=1cm ]{geometry}

%%% Кодировки и шрифты %%%
\usepackage{metalogo}
\usepackage{cmap}						% Улучшенный поиск русских слов в полученном pdf-файле
%\usepackage[T2A]{fontenc}				% Поддержка русских букв
%\usepackage[utf8]{inputenc}				% Кодировка utf8
\usepackage{textcomp}
\usepackage[english, russian]{babel}	% Языки: русский, английский
%\usepackage[14pt]{extsizes}

\usepackage{xltxtra}
\usepackage{xunicode}
\usepackage{color}
\usepackage{colortbl}

\usepackage{multicol}

%%%%%%%%%
 \usepackage{tikz}
 \usepackage{pgf,pgfarrows,pgfnodes,pgfplots}
 \usepackage{pgfplots}
 \makeatletter
 \pgfplotsset{compat=newest}
 \makeatother
\usepgfplotslibrary{external,patchplots}
\usetikzlibrary{%
  arrows,%
  shapes.misc,% wg. rounded rectangle
  shapes.arrows,%
  shapes.geometric,%
  chains,%
  matrix,%
  positioning,% wg. " of "
  scopes,%
  decorations.pathmorphing,% /pgf/decoration/random steps | erste Graphik
  shadows%
}

\usepackage{listings}
\usepackage{verbatim}


%\usepackage{pscyr}						% Красивые русские шрифты

%%% Математические пакеты %%%
\usepackage{amsfonts,amssymb,amscd} % Математические дополнения от AMS

%%% Оформление абзацев %%%
\usepackage{indentfirst} % Красная строка

%%% Цвета %%%
%\usepackage[usenames]{color}
%\usepackage{color}

%%% Таблицы %%%
\usepackage{tabularx}
\usepackage{longtable}					% Длинные таблицы
\usepackage{multirow,makecell,array}	% Улучшенное форматирование таблиц

%%% Общее форматирование
\usepackage[singlelinecheck=off,center]{caption}	% Многострочные подписи
\usepackage{soul}									% Поддержка переносоустойчивых подчёркиваний и зачёркиваний

%%% Библиография %%%
%\usepackage{cite} % Красивые ссылки на литературу

%%% Гиперссылки %%%
%\usepackage[linktocpage=true,plainpages=false,pdfpagelabels=false]{hyperref}

%%% Изображения %%%
\usepackage{graphicx} % Подключаем пакет работы с графикой

%%% Оглавление %%%
%\usepackage[subfigure]{tocloft}
\usepackage{pdftexcmds}
%\usepackage[svgpath=fig/]{svg}
\usepackage{float}
\usepackage{wrapfig}
\usepackage{sidecap}

\usepackage{chemfig}


\usepackage{arydshln}

%%%% Notes
\usepackage{pgfpages}
%\setbeameroption{show notes on second screen}
%%%% EndNotes
%\usepackage{subfigure}

\newcolumntype{P}[1]{>{\raggedright\arraybackslash}p{#1}}
\newcolumntype{Z}[1]{>{\raggedleft\arraybackslash}p{#1}}
\newcolumntype{Y}[1]{>{\centering\arraybackslash}p{#1}}
\newcommand{\tss}[1]{\textsuperscript{#1}}
\newcommand{\tbs}[1]{\textsubscript{#1}}
\newcommand{\dg}{\ensuremath{^\circ}}

\newcommand{\paral}[2]{%
    \begin{tikzpicture}[scale=#2]%
          \draw [ color=black, thick, fill=#1 ] (0,0) %
    --(1,0)--(1.5,1)--(0.5,1)-- (0,0);%
          \end{tikzpicture}
            }
\newcommand{\romb}[2]{%
    \begin{tikzpicture}[scale=#2]%
          \draw [ color=black, thick, fill=#1 ] (0,0) %
    --(1,0)--(1.5,0.866)--(1,1.732)--(0,1.732)--(-0.5,0.866)  -- (0,0);%
          \end{tikzpicture}
            }
\newcommand{\wat}[1]{%
\begin{tikzpicture}[xscale=#1,yscale=#1]
        \draw [ color   = black,   thick,   ] (1,2) -- (2,1);
        \draw [ color   = black,   thick,   ] (3,2) -- (2,1);
        \draw[fill=white] (1,2) circle [radius=.3];  
        \draw[fill=white] (3,2) circle [radius=.3];  
        \draw[fill=red!80] (2,1) circle [radius=.3];  
\end{tikzpicture}
}
\newcommand{\disk}[1]{%
   \begin{tikzpicture}\draw[fill=#1,overlay,#1](0,0)circle(2pt);
   \end{tikzpicture} 
}
\newcommand{\ddisk}[2]{%
\begin{tikzpicture}\draw[fill=#1,overlay,#1] (0,#2pt) circle (#2pt);
\end{tikzpicture} 
}

\newcommand{\gua}[2]{%
%         \setatomsep{2.0em}\setcrambond{0.2em}{}{}%
         \chemfig[atom sep=1.5em,cram width=0.2em]{%
         \ddisk{magenta}{4}-[:#2]\disk{blue}*5(-=@{n7#1}\disk{blue}-(*6(-(=@{o#1}\disk{red})-\disk{blue}(-@{n#1}\disk{gray})-(%
         -\disk{blue}(-[::60]\disk{gray})(-[::-60]@{n2#1}\disk{gray}))=\disk{blue}-))=-)%     
               }}
\newcommand{\includegraphicsfs}[1]{%
        \begin{tikzpicture}[remember picture,overlay]
                \node[at=(current page.center),fill=white] {
                 \includegraphics[width=\paperwidth]{#1}
                   };
        \end{tikzpicture}
  }

%\usepackage{luacode}
%\begin{luacode*}
%function printtable(k)
%     n=string.len(k)
%     for i=1,n do 
%      for j=1,n do 
%        if (string.sub(k,i,i) == "G" and string.sub(k,j,j) == "C") or 
%        (string.sub(k,i,i) == "C" and string.sub(k,j,j) == "G") then  
%        tex.sprint("\\node at (".. i/5 ..",".. j/5 ..") {".. 3 .."};" )
%        end
%        if (string.sub(k,i,i) == "A" and string.sub(k,j,j) == "U") or 
%        (string.sub(k,i,i) == "U" and string.sub(k,j,j) == "A") then  
%        tex.sprint("\\node at (".. i/5 ..",".. j/5 ..") {"..  2 .."};" )
%        end
%       end
%     end 
%     tex.print("\\draw[blue,thick] (0.2,".. n/5 ..") -- (".. n/5+0.2 ..",0);")
%end
%\end{luacode*}
%\begin{luacode*}
%function printtableold(k)
%%     n=string.len(k)
%     for i=1,n do 
%      for j=1,n do 
%        if (string.sub(k,i,i) == "G" and string.sub(k,j,j) == "C") or 
%        (string.sub(k,i,i) == "C" and string.sub(k,j,j) == "G") then  
%        tex.sprint("\\node at (".. i/5 ..",".. j/5 ..") {".. string.sub(k,i,i).."};" )
%        end
%        if (string.sub(k,i,i) == "A" and string.sub(k,j,j) == "U") or 
%        (string.sub(k,i,i) == "U" and string.sub(k,j,j) == "A") then  
%        tex.sprint("\\node at (".. i/5 ..",".. j/5 ..") {".. string.sub(k,i,i).."};" )
%        end
%       end
%     end 
%     tex.print("\\draw[blue,thick] (0.2,".. n/5 ..") -- (".. n/5+0.2 ..",0);")
%end
%\end{luacode*}
%
%
%
%
%
%
