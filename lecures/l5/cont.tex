
\begin{frame}
[plain]
  \titlepage
\end{frame}
\section{Введение}
\begin{frame}
{Дизайн остова или формы белка}{}
 \begin{itemize}
  \item
      \textbf{Локальный дизайн:} Вставки и делеции для достижения необходимой формы локального окружения
\vspace{0.2cm}
  \item
	  \textbf{Глобальный дизайн формы:} Подгон последовательности под фолд и предсказание структуры из последовательности, 
\vspace{0.2cm}
 \end{itemize}
\end{frame}
\begin{frame}{Основные проблемы:}{}
 \begin{itemize}
  \item
	  Монте-Карло: 100 а.к. 3\textsuperscript{N} степеней свободы, получаем 10\textsuperscript{48} конформаций.
\vspace{0.2cm}
  \item
	  \textbf{Парадокс Левинталя:} "Промежуток времени, за который полипептид приходит к своему скрученному состоянию, на много порядков меньше, чем если бы полипептид просто перебирал все возможные конфигурации".
\vspace{0.2cm}
  \item
Для решения разумно использовать накопленные знания для моделирования.
 \end{itemize}
\end{frame}
\section{Сравнительное моделирование}
\begin{frame}{Построение остова}

\begin{itemize}
	\item
Генерируем координаты остова желаемого белка из  каких-то элементов.
\item
Не обязательно использовать координаты, могут подойти дистанционные ограничения.
\item
На основании ограничений можно  построить структуру для polyA
\end{itemize}
\end{frame}
\begin{frame}{Моделирование петель}

	\begin{itemize}
		\item
			Эмпирическое моделирование:
			\begin{itemize}
				\item
					Поиск подходящего фрагмента по PDB 
				\item
					Использовать базы данных (LIP, etc..)
			\end{itemize}
 		         \item
					 Молекулярная механика.
				 \item
					 Монте-Карло.
				 \item
					 Rosetta:
					 \begin{itemize}
						 \item
			 Поиск фрагментов близких по последовательности.
						 \item
			 Комбинирование результатов поиска с помощью Монте-Карло.
					 \end{itemize}
					 \vspace{1.0cm}
		  Комбинации выше перечисленных.
	\end{itemize}
\end{frame}

\begin{frame}{Низкое разрешение в Rosetta}
  \centering       
  \begin{tikzpicture}%
   \node (2) at (4,2) {%
     \footnotesize
    \setchemfig{atom sep=1.7em, cram width=0.2em} {}{}%
     \chemfig{H{_2}N-[1](-[2]-[1]-[3](=[4]O)-[1]O{^-})-[7]C(=[6]O)-[1]N
         (-[2]H)-[7](-[6](-[7])-[5])-[1]C(=[2]O)-[7]N(-[6]H)
          -[1](-[2]-[1]**6(------))-[7]C(=[6]O)-[1]OH}};
     \node (d1) at (3,3) {\ddisk{orange}{2}};   
     \node (d2) at (3.85,0.5) {\ddisk{orange}{2}};   
     \node (d3) at (5,2.5) {\ddisk{orange}{2}};   
  \end{tikzpicture}
\end{frame}

\begin{frame}{Моделирование петель, CCD}
    \begin{tikzpicture}
     \node [rectangle, fill = red!50!black, anchor=center, minimum width=0.1\paperwidth, minimum height=1.0cm] (1) at  (0,0)
     { Последовательность петли   };
     \node [rectangle, fill = orange!50!black,  anchor=center, minimum width=0.1\paperwidth, minimum height=1.0cm, below of=1, yshift=-0.5cm,xshift=0cm] (2) 
     { CG ,  Построение петли из фрагментов};
     \node [rectangle, fill = yellow!50!black,  anchor=center, minimum width=0.1\paperwidth, minimum height=1.0cm, below of=2, yshift=-0.5cm,xshift=0cm] (3) 
     { $\Phi,\Psi$  генерация состояний при фиксировании одного конца};
     \node [rectangle, fill = green!50!black,  anchor=center, minimum width=0.1\paperwidth, minimum height=1.0cm, below of=3, yshift=-0.5cm,xshift=0cm] (4) 
     { Отбор  состояний закрывающих петлю};
     \node [rectangle, fill = blue!50!black,  anchor=center, minimum width=0.1\paperwidth, minimum height=1.0cm, right of=4, yshift=-0.0cm,xshift=6cm] (5) 
     { Оптимизация };
     \draw[-stealth] (1.south) -- (2.north);
     \draw[-stealth] (2.south) -- (3.north);
     \draw[-stealth] (3.south) -- (4.north);
     \draw[-stealth] (4.east) -- (5.west);
    \end{tikzpicture}
\end{frame}

\begin{frame}{Моделирование петель, GeneralizedKIC}
    \begin{tikzpicture}
     \node [rectangle, anchor=center, minimum width=0.1\paperwidth, minimum height=1.0cm] (1) at  (0,0)
     {\includegraphics[height=.7\textheight]{GenKICoverviewsm.png} };
     \node [rectangle, fill = orange!50!black,  anchor=center, minimum width=0.1\paperwidth, minimum height=1.0cm, right of=1, yshift=2.5cm,xshift=6cm] (2) 
     {Делим петлю на две части };
     \node [rectangle, fill = yellow!50!black,  anchor=center, minimum width=0.1\paperwidth, minimum height=1.0cm, below of=2, yshift=-1.2cm,xshift=0cm] (3) 
     { Меняем $\Phi,\Psi$ в фрагментах };
     \node [rectangle, fill = green!50!black,  anchor=center, minimum width=0.1\paperwidth, minimum height=1.0cm, below of=3, yshift=-1.2cm,xshift=0cm] (4) 
     { Оптимизируем фрагменты для закрытия петли};
    \end{tikzpicture}
\end{frame}


\begin{frame}{Поиск последовательности }
    FastDesign
    \begin{itemize}
        \item Перебор ротамеров в 4 раундах
        \item Оптимизация геометрии
        \item 12 раундов ротамеры/оптимизация
        \item Разделение позиций : ‘core’, ‘boundary’,  ‘surface
        \item 80000 вариантов для топологии и сортировка по  энергии
    \end{itemize}
\end{frame}


\begin{frame}{Дизайн циклических пептидов}
    \centering
		\includegraphics[height=0.8\textheight]{ss-variants}  
\end{frame}


\begin{frame}{дизайн циклических пептидов}
		\includegraphics[height=0.8\textheight]{ss-variants-results }  
\end{frame}


\begin{frame}{Дизайн по варианту укладки с AlphaFold}
    \centering
		\includegraphics[height=0.8\textheight]{alpha-design-1}  
\end{frame}

\begin{frame}{Дизайн по варианту укладки с AlphaFold}
    \centering
		\includegraphics[height=0.4\textheight]{alpha-design-2}  
\end{frame}


\begin{frame}{Словами}
        \begin{itemize}
            \item Два модуля оптимизации последовательности GD и МСМС (Марковские цепи в Монте-Карло)
            \item В модуле  GD дизайн начинается с последовательностей характерных  для SS
            \item "Обратный проброс ошибки" для структуры из дизайна от желаемой и оптимизация PSSM
            \item После нескольких раундов последовательность "мутируется" MCMC
            \end{itemize}        
\end{frame}
                


\begin{frame}{Результат Alpha Design}
    \centering
		\includegraphics[height=0.8\textheight]{alpha-design-3}  
\end{frame}

\begin{frame}{Результат Alpha Design}
    \centering
		\includegraphics[height=0.4\textheight]{alpha-design-4}  
\end{frame}



\begin{frame}{"network hallucination"}
    \centering
    \includegraphics[width=\textwidth]{halluc}
\end{frame}

\begin{frame}{"network hallucination"}
    \centering
    \includegraphics[width=\textwidth]{halluc-mcmc}
\end{frame}


\begin{frame}{"network hallucination"}
    \centering
    \includegraphics[width=\textwidth]{halluc-mcmc}
\end{frame}


\begin{frame}{Галлюцинации идеальных  белков}
    \centering
    \includegraphics[width=\textwidth]{halluc-backbone-1}
\end{frame}

\begin{frame}{Галлюцинации идеальных  белков}
    \centering
    \includegraphics[width=0.6\textwidth]{halluc-backbone-2}
\end{frame}

\begin{frame}{Галлюцинации идеальных  белков}
    \centering
    \includegraphics[width=\textwidth]{halluc-backbone-3}
    129 белков проверено - 27 кластеров
\end{frame}


\begin{frame}{Заключение}
        \begin{itemize}
            \item Из модели для нативных получили новые белки
            \item Высокая точность совпадения предсказаний и эксперимента
            \item Регулярность структур, короткие петли, они идеальны :)
            \item Функция потерь может включать в себя желания пользователя (прошлая лекция)
            \end{itemize}        
\end{frame}



\begin{frame}{Диффузия}
    \centering
    \includegraphics[width=\textwidth]{diff-1}
\end{frame}

\begin{frame}{Диффузия}
    \centering
    \includegraphics[width=\textwidth]{diff-2}
\end{frame}

\begin{frame}{Диффузия}
    \centering
    \includegraphics[width=\textwidth]{diff-3}
     На каждом  шаге t модель берет X(t+1), а затем прогнозирует обновленную структуру X(0). Следующий ввод координат X(t-1) в модель генерируется
     путем интерполяции с шумом в направлении X(0).
\end{frame}


\begin{frame}{Диффузия, без условий}
    \centering
    \includegraphics[width=\textwidth]{diff-5}
\end{frame}

\begin{frame}{Диффузия, с условиями }
    \centering
    \includegraphics[height=0.8\textheight]{diff-6}
\end{frame}


\begin{frame}{Призводительность}
        \begin{itemize}
            \item RF-диффузия значительно превосходит (AF2 score) галлюцинацию с RF
            \item Генерация RF-диффузии также более эффективна в вычислениях, чем неограниченная галлюцинация с RF
            \item Белок из 100 остатков может быть создан всего за 11 секунд на графическом процессоре NVIDIA RTX A4000, в отличие от RF Hallucination, который занимает около 8,5 минут.
            \end{itemize}        
\end{frame}

\begin{frame}{Диффузия, с условиями }
    \centering
    \includegraphics[width=\textwidth]{diff-7}
\end{frame}
\begin{frame}{Диффузия, с условиями }
    \centering
    \includegraphics[width=\textwidth]{diff-8}
\end{frame}

\begin{frame}{Дизайн "байндера"}
    \centering
    \includegraphics[width=\textwidth]{diff-9}
\end{frame}
			
%\begin{frame}
%{Функция не сдачи праков:}
%	\begin{center}
%		\input{/tmp/jopa.tex}
%    \setlength{\fboxsep}{1pt}
%     \fcolorbox{black}{white}{ 
%		 \includegraphics[width=1.0\textheight]{/tmp/jopa.png}
%	 }
%	\end{center}
%\end{frame}



			 
	


