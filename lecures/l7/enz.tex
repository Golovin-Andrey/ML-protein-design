    
\begin{frame}[plain]
  \titlepage
\end{frame}


\begin{frame}{Ферменты это биокатализаторы}
Катализатор – вещество, ускоряющее химическую реакцию. Биокатализатор – такое вещество биологической природы (белок, нуклеиновая кислота).
\begin{columns}
\begin{column}{0.7\textwidth}
    \includegraphics[width=1\textwidth]{enz-chor-scheme.png}
\end{column}
\begin{column}{0.3\textwidth}
    Фермент хоризмат мутаза ускоряет эту реакцию в 10\tss{6} раз
\end{column}
\end{columns}
\end{frame}

\begin{frame}{Биокатализаторы в промышленности и медицине}
    \small
\begin{columns}
\begin{column}{0.3\textwidth}
    Пищевая индустрия:
    \begin{itemize}
        \item Ферментация чего угодно
        \item Разрушение глютена
        \item Осветление напитков
        \item Безлактозное молоко
        \item Синтез ароматизаторов
        \item Удаление неприятных ароматов
    \end{itemize}
    \end{column}
\begin{column}{0.3\textwidth}
    Не пищевая индустрия:
    \begin{itemize}
        \item Производство биотоплива
        \item Деградация пластиков
        \item Очистка сточных вод
        \item Улучшение тканей
        \item Стиральные порошки
        \item Зубные пасты
        \item Производство бумаги
        \item Производство чистых аминокислот
    \end{itemize}
\end{column}
\begin{column}{0.3\textwidth}
    Медицина:
    \begin{itemize}
        \item Синтез полусинтетических препаратов
        \item Синтез прекурсоров
        \item Терапевтические ферменты
        \end{itemize}
\end{column}
\end{columns}
\end{frame}



\includegraphicsfsh{enz-factors.png}


\begin{frame}{Как они это делают}
    \begin{tikzpicture}
        % \node[circle,draw,minimum size=0cm,fill=purple!80] (b1) at (0,10) {};
        \node (1) at (0,0)  {
            \includegraphics[width=.8\textwidth]{enz-chor-tr3d.png}};
        \node [text width=3cm ] (2)  at (-6,0) {Стабилизация переходного \\ состояния остатками хоризмат мутазы};
        \node (3) at (4,0) {
             \includegraphics[width=.3\textwidth]{enz-chor-scheme.png}
        };
    \end{tikzpicture}
\end{frame}



\begin{frame}[plain]
\begin{columns}
\begin{column}{0.3\textwidth}
            \includegraphics[width=1\textwidth]{enz-chor-w1.png}
            \includegraphics[width=1\textwidth]{enz-chor-e1.png}
\end{column}
\begin{column}{0.3\textwidth}
            \includegraphics[width=1\textwidth]{enz-chor-w2.png}
            \includegraphics[width=1\textwidth]{enz-chor-e2.png}
\end{column}
\begin{column}{0.3\textwidth}
            \includegraphics[width=1\textwidth]{enz-chor-w3.png}
            \includegraphics[width=1\textwidth]{enz-chor-e3.png}
\end{column}
\end{columns}
\end{frame}


\begin{frame}{Как они это делают}
            \includegraphics[width=.7\textwidth]{enz-scheme.png}
\end{frame}

\begin{frame}{Чаще всего все сложнее}
\begin{columns}
\begin{column}{0.7\textwidth}
    \includegraphics[height=.85\textheight]{enz-scheme-mult.png}
\end{column}
\begin{column}{0.3\textwidth}
    Многостадийный катализ сериновой протеазы с ковалентными конъюгатами субстрат-фермент
\end{column}
\end{columns}
\end{frame}


\begin{frame}{Чаще всего все сложнее}
\begin{columns}
\begin{column}{0.7\textwidth}
    \includegraphics[height=.85\textheight]{enz-scheme-mult2.png}
\end{column}
\begin{column}{0.3\textwidth}
    Многостадийный катализ пируват карбоксилазы с кофактором-катионом металла и коферментом-биотином
\end{column}
\end{columns}
\end{frame}



\begin{frame}{Структурные особенности}
\begin{columns}
\begin{column}{0.5\textwidth}
    \includegraphics[width=1\textwidth]{enz-3d-1.png}
\end{column}
\begin{column}{0.5\textwidth}
    Непосредственно в реакцию вовлечены только 1-3 остатка. Они называются активным центром фермента.
    \newline
   Конкретно в этом ферменте в реакции участвует только этот аспартат
\end{column}
\end{columns}
\end{frame}

\begin{frame}{Структурные особенности}
\begin{columns}
\begin{column}{0.5\textwidth}
    \includegraphics[width=1\textwidth]{enz-3d-2.png}
\end{column}
\begin{column}{0.5\textwidth}
    Остатки активного центра непосредственно участвуют в реакции с реагентами
\end{column}
\end{columns}
\end{frame}


\begin{frame}{Структурные особенности}
\begin{columns}
\begin{column}{0.5\textwidth}
    \includegraphics[width=1\textwidth]{enz-3d-2.png}
\end{column}
\begin{column}{0.5\textwidth}
    Остатки активного центра непосредственно участвуют в реакции с реагентами\\
    Зачем нужен весь остальной белок?\\
    \vspace{1cm}
Для начала реакции нам надо поймать 
и зафиксировать реагенты в правильном положении \\

Остатки, которые это делают - это карман связывания
\end{column}
\end{columns}
\end{frame}
 


\begin{frame}{Активный центр и карманы связывания протеазы}
\begin{columns}
\begin{column}{0.5\textwidth}
    \includegraphics[width=1\textwidth]{enz-3d-protease.png}
\end{column}
\begin{column}{0.5\textwidth}\
    Без триады не будет реакции
Какой именно пептид будет разрезан, определяют структуры карманов P1-P3, P1’-P3’
\end{column}
\end{columns}
\end{frame}


\begin{frame}{Структурные особенности}
\begin{columns}
\begin{column}{0.5\textwidth}
    \includegraphics[height=.85\textheight]{enz-3d-module.png}
\end{column}
\begin{column}{0.5\textwidth}
    Модульность: 9 эукариотических киназ (мутации всех замешаны в онкологии). 
\begin{itemize}
    \item Активные сайты одинаковые
    \item Карманы связывания АТФ одинаковые
    \item Открытие/закрытие одинаковое
    \item Карманы связывания субстратов разные
\end{itemize}
\end{column}
\end{columns}
\end{frame}


\begin{frame}{Дизайн и его валидация}
\begin{columns}
\begin{column}{0.5\textwidth}
    Дизайн – создание объекта
с заданными функциями
\begin{itemize}
    \item Создание белка
    \item Создание гена
    \item Создание сиквенса
\end{itemize}        
\textbf{
Инструменты дизайна – то, что делает создание сиквенса не случайным
}
\end{column}
\begin{column}{0.5\textwidth}
    Имеем сиквенс, выполняет ли соответствующий объект функцию?
\begin{itemize}
    \item Сразу в экспериментальный assay
    \item Имитация эксперимента – молекулярное моделирование/ML
\end{itemize}
\textbf{Инструмент валидации – то, что сопоставляет дизайну оценку его функции (с некоторой ошибкой)}
\end{column}
\end{columns}
\end{frame}

\begin{frame}{Дизайн и его валидация}
\textbf{Тривиальный вычислительный дизайн}\\
Генерация случайных гипотез и их вычислительная валидация
\begin{itemize}
    \item Вычислительная валидация хоть и дешевле экспериментальной, но тоже трудозатратна (при отлаженном и предсказуемом пайплайне валидация процессивности одного варианта фермента – неделя на современной рабочей станции)
    \item Нужно генерировать гипотезы не случайно и иметь многоуровневый процесс валидации от дешевого и неточного этапа до дорогого и точного
Создание белка
\end{itemize}        
\end{frame}

\begin{frame}{Задачи дизайна ферментов}
\begin{columns}
\begin{column}{0.5\textwidth}
    \textbf{(До)-дизайн (тюнинг)}

    Уже есть стартовый вариант, он качественно делает то, что нужно.
Хотим количественного улучшения
\end{column}
\begin{column}{0.5\textwidth}
\begin{itemize}
    \item \textbf{Редизайн} : Есть стартовый вариант, 
он делает что-то похожее, 
но не совсем то, что нужно
    \item \textbf{Де-ново дизайн} Нет никакого стартового варианта
\end{itemize}
\end{column}
\end{columns}
\end{frame}


\begin{frame}{Тюнинг}
\begin{columns}
\begin{column}{0.5\textwidth}
    \begin{itemize}
        \item Стабильность
        \item Растворимость
        \item pH-зависимость стабильности
    \end{itemize}
\end{column}
\begin{column}{0.5\textwidth}
\begin{itemize}
    \item pH-оптимум активности
    \item Улучшение аффинности связывания
    \item Улучшение константы реакции
\end{itemize}
\end{column}
\end{columns}
\end{frame}

\begin{frame}{pH-зависимость стабильности}
\begin{columns}
\begin{column}{0.7\textwidth}
    \includegraphics[width=1\textwidth]{enz-ph-beta.png}
\end{column}
\begin{column}{0.3\textwidth}
    В пенициллин ацилазе в щелочных условиях разрушается водородная связь между E482 и D484. Замена D484N повысила стабильность в щелочной среде в 9 раз.
\end{column}
\end{columns}
\end{frame}


\begin{frame}{Аффинность}
    Задача ничем не отличается от улучшения неферментативного кармана связывания (за исключением строгой немутабельности каталитических остатков).
\begin{itemize}
    \item \textbf{Как решаем:} MC перебор с помощью Rosetta
    \item \textbf{Нюанс:} Для ферментов такая задача ставится редко. Гораздо выгоднее понизить барьер активации.
\end{itemize}
\end{frame}


\includegraphicsfsh{enz-gluten.png}

\begin{frame}{Оптимизация кумамолизина }
\begin{columns}
\begin{column}{0.5\textwidth}
    \includegraphics[height=.85\textheight]{enz-gluten-2.png}
\end{column}
\begin{column}{0.5\textwidth}
    Стартовая идея о том, как должен располагаться нужный субстрат. \\
Раунды MC перебора остатков фермента вокруг. \\
Допуск небольшой подвижности остова улучшает результаты, но сильно дороже вычислительно.
\end{column}
\end{columns}
\end{frame}


\begin{frame}{Константа}
    \begin{itemize}
        \item Стабилизация переходного состояния – дизайн непосредственно контактирующих остатков 
        \item Оптимизация путей перераспределения заряда – дизайн электростатического поля, может достигаться заменами в остатках на удалении от субстрата
        \item Тривиальный дизайн с быстрой валидацией (замена вносится в состояние реагентов, моделируется реакция с оценкой барьера)
    \end{itemize}
    \includegraphics[height=.3\textheight]{enz-barier.png}
\end{frame}

\begin{frame}{Стратегия 1}
    Стратегия 1 крайне редко используется для тюнинга
    \begin{itemize}
        \item природные сайты уже максимально оптимизированы эволюционно
        \item даже если есть теоретическая возможность улучшения, ее размер меньше типичной ошибки силового поля Rosetta
    \end{itemize}
    Где используется – редизайн и де-ново дизайн (обсудим дальше).
\end{frame}

\begin{frame}{Стратегия 2}
\begin{columns}
\begin{column}{0.7\textwidth}
    \includegraphics[width=1\textwidth]{enz-lig-field.png}
\end{column}
\begin{column}{0.3\textwidth}
    Стратегия “снизу-вверх”. Попытка реконструировать необходимое поле от структуры переходного состояния или от пути реакции.


Предложена недавно, показала применимость на модельных системах. Однако пока не применялась для создания нового продукта.
\end{column}
\end{columns}
\end{frame}

\begin{frame}{Стратегия 3}
\begin{columns}
\begin{column}{0.7\textwidth}
    \includegraphics[width=1\textwidth]{enz-qmmm-val.png}
\end{column}
\begin{column}{0.3\textwidth}
    Гипотезы о вариантах: рациональный ручной выбор

Валидация: QM/MM метадинамика
\end{column}
\end{columns}
\end{frame}

\begin{frame}{Стратегия 3: CADEE}
\begin{columns}
\begin{column}{0.4\textwidth}
    \includegraphics[height=.85\textheight]{enz-cadee.png}
\end{column}
\begin{column}{0.4\textwidth}
    Гипотезы о вариантах: Ala-сканирование для выявления хотспотов с вычислением барьера с помощью EVB, мутация на все, снова вычисление барьера с помощью EVB

Валидация: вшита в генерацию гипотез (EVB)
\end{column}
\end{columns}
\end{frame}


\begin{frame}{Редизайн}
    \Large
\begin{itemize}
    \item \textbf{Как решаем:} Кофакторная специфичность
    \item \textbf{Как решаем:} Субстратная специфичность
    \item \textbf{Как решаем:} Предпочтительный механизм
\end{itemize}
\end{frame}

\begin{frame}{Кофакторная специфичность}
\begin{columns}
\begin{column}{0.5\textwidth}
    \includegraphics[width=1\textwidth]{enz-cofactor-sp.png}
\end{column}
\begin{column}{0.5\textwidth}
    Концептуально то же самое, что дизайн кармана связывания\\
Кейс: фермент работает с НАДФ, а мы хотим заставить работать с НАД, так как в клетке его больше. Или наоборот.\\
По меньшей мере 33 успешные работы к данному моменту для пары НАД/НАДФ. Для чего-то еще – исчезающие количества.
\end{column}
\end{columns}
\end{frame}


\begin{frame}{Субстратная специфичность}
    \includegraphics[width=.3\textwidth]{enz-substrate-sp.png}\\
Пример: тирозин-фосфатаза снимает фосфат с тирозина. Хотим, чтобы снимала с инозитол-фосфата. Реакция одна и та же – нужно добиться хорошего
позиционирования нового субстрата.\\
Что делали: Rosetta \\
Что получили: дизайны с kcat/Km на 1 порядок хуже, чем для пары дикий тип-фосфотирозин (Km при этом лучше, kcat сильно хуже)\\
\end{frame}


\begin{frame}[plain]
\includegraphics[height=\textheight]{anz-apartase.png}
\end{frame}
\begin{frame}[plain]
\includegraphics[height=\textheight]{anz-apartase2.png}
\end{frame}


\begin{frame}{Региоспецифичность}
\begin{columns}
\begin{column}{0.3\textwidth}
    \includegraphics[width=1\textwidth]{enz-regio1.png}
\end{column}
\begin{column}{0.7\textwidth}
    Кейс: энантиоселективные варианты эпоксид гидролазы \\
    \includegraphics[width=1\textwidth]{enz-regio2.png}\\
    Для каждого варианта нужно обеспечить связывание, приводящее к предпочтительной атаке по нужному атому углерода. Положение атакующей воды лучше не
    менять, ибо оно уже оптимизировано для низких барьеров.\\
Как: Rosetta + валидация дизайнов короткими MD 
\end{column}
\end{columns}
\end{frame}


\begin{frame}{Дизайн субстратной специфичности от структуры TS}
\begin{columns}
\begin{column}{0.5\textwidth}
    Можем заглянуть немного в будущее
и оптимизировать непосредственно связывание переходного состояния
под версию субстрата. \\
Пример: заставить метионил-тРНК синтетазу пришивать вместо метионина что-то другое. \\
Как: Proteus (Wang-Landau MC). 
\end{column}
\begin{column}{0.5\textwidth}
    \includegraphics[width=0.48\textwidth]{enz-subst-ts1.png}
    \includegraphics[width=0.48\textwidth]{enz-subst-ts2.png}\\
    \footnotesize  В MC мы ориентируемся на скор, если он улучшается, мы принимаем изменение. Положительное изменение скора не обязательно значит
    образование взаимодействий с лигандом/TS. Можем сделать предварительный MC скан на apo-ферменте и сконструировать поправку к скору, делающую все
    замены равнозначными, чтобы отбирать только те замены, которые повышают скор исключительно благодаря взаимодействиям с лигандом/TS.
\end{column}
\end{columns}
\end{frame}

\begin{frame}[plain]
    \centering
    \Huge{Де-ново дизайн}\\
    \Large{Новая молекула с заданной ферментативной функцией}
\end{frame}

\begin{frame}{Переходное состояние как лиганд}
\begin{columns}
\begin{column}{0.5\textwidth}
    Допустим, мы уже умеем дизайнить карманы связывания под малые молекулы. Нам нужно
    \begin{itemize}
        \item Описать переходное состояние субстрата как малую молекулу
        \item Зафиксировать непосредственно участвующие в реакции остатки
        \item Остатки + переходное состояние = теозим
    \end{itemize}
    Задача сводится к поиску места для размещения такого теозима в т.н. структурных матрицах (скэффолдах)
\end{column}
\begin{column}{0.5\textwidth}
    Варианты теозима для этой реакции: \\
    \includegraphics[width=1\textwidth]{enz-denovo-1.png}\\
\end{column}
\end{columns}
\end{frame}


\begin{frame}{Откуда брать скэффолды?}
\begin{columns}
\begin{column}{0.7\textwidth}
    Допустим, мы уже умеем дизайнить карманы связывания под малые молекулы. Нам нужно
    \footnotesize
    \begin{itemize}
        \item Весь PDB
        \item Весь метапротеом + AlphaFold2
        \item Искусственные скэффолды \newline Пока умеем делать что-то простое: фолды из элементов надвторичной структуры, простые альфа-пучки, бета-бочонки
        \item Предковая реконструкция \newline Идея: современные ферменты – оптимизированные специалисты. Если начнем что-то менять, скорее всего сломаем. Но они произошли от предка-генералиста. Попробуем предсказать его последовательность, получим структуру и возьмем ее за старт наших дизайнов. Описать переходное состояние субстрата как малую молекулу
    \end{itemize}
    Хотелось бы воображать скэффолд под конкретный активный центр, но этого пока даже близко нет
\end{column}
\begin{column}{0.3\textwidth}
    \includegraphics[width=1\textwidth]{enz-denovo-2.png}
\end{column}
\end{columns}
\end{frame}


\begin{frame}{Дизайн ферментов: стабилизация TS}
    \footnotesize
\begin{columns}
\begin{column}{0.5\textwidth}
    Удалось сделать несколько де-ново ферментов (для реакций, не имеющих своего фермента в природе): Кемп-Элиминаза, Дильс-Альдераза
    \begin{itemize}
        \item Только 1 стадия
        \item Любое число > 0 = успех
        \item Структура TS получена из КМ моделирования Весь PDB
    \end{itemize}
\end{column}
\begin{column}{0.3\textwidth}
    Не удается улучшать уже имеющиеся ферменты
    \begin{itemize}
        \item Ошибка в вычислении скора выше, чем ожидаемый результат
        \item Имеющиеся ферменты это уже итог долгой эволюции – простые решения (например, единичные замены) исчерпаны. Требуется масштабный редизайн
        \item Очень просто сломать
    \end{itemize}
\end{column}
\end{columns}
Не известны работы по успешному де-ново дизайну ферментов для сложных реакций (больше 1 стадии, наличие кофакторов/коферментов)
\end{frame}


\begin{frame}{RF-design}
    \centering
    \includegraphics[height=1\textheight]{enz-rfdesig.png}\\
\end{frame}

\begin{frame}{Tricks}
    \begin{itemize}
        \item \textbf{Неудачно}: градиентный спуск путем обратного распространения ошибки через AF, плохо ложились боковые цепи
        \item \textbf{Лучше}: Двухэтапный подход с использованием как AF, так и trRosetta (для сглаживания ландшафта loss функции) и описания активного сайта
            на уровне остова. Второй этап это оптимизация последовательности с AF, при сохранении хода остова (fixed bb design)
        \end{itemize}
\end{frame}

\begin{frame}{RF-diffusion}
    \centering
    \includegraphics[width=1\textwidth]{enz-rfdiffusion.png}\\
\end{frame}

\begin{frame}{External potential}
    \centering
    \includegraphics[width=1\textwidth]{enz-rf-potentials.png}\\
\end{frame}

\includegraphicsfs{enz-final.png}
