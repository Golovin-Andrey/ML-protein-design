\documentclass[professionalfont,table,svgnames,xcolor=svgnames,aspectratio=169]{beamer}
%\pdfpageattr {/Group << /S /Transparency /I true /CS /DeviceRGB>>}
\usepackage{multimedia}
\usepackage{hyperref}
%%% Поля и разметка страницы %%%
\usepackage{lscape}		% Для включения альбомных страниц
\usepackage{geometry}	% Для последующего задания полей
%\usepackage[a4paper, mag=1000,left=2.5cm, right=1cm, top=2cm, bottom=2cm, headsep=0.7cm, footskip=1cm ]{geometry}

%%% Кодировки и шрифты %%%
\usepackage{metalogo}
\usepackage{cmap}						% Улучшенный поиск русских слов в полученном pdf-файле
%\usepackage[T2A]{fontenc}				% Поддержка русских букв
%\usepackage[utf8]{inputenc}				% Кодировка utf8
\usepackage{textcomp}
\usepackage[english, russian]{babel}	% Языки: русский, английский
%\usepackage[14pt]{extsizes}

\usepackage{xltxtra}
\usepackage{xunicode}
\usepackage{color}
\usepackage{colortbl}

\usepackage{multicol}

%%%%%%%%%
 \usepackage{tikz}
 \usepackage{pgf,pgfarrows,pgfnodes,pgfplots}
 \usepackage{pgfplots}
 \makeatletter
 \pgfplotsset{compat=newest}
 \makeatother
\usepgfplotslibrary{external,patchplots}
\usetikzlibrary{%
  arrows,%
  shapes.misc,% wg. rounded rectangle
  shapes.arrows,%
  shapes.geometric,%
  chains,%
  matrix,%
  positioning,% wg. " of "
  scopes,%
  decorations.pathmorphing,% /pgf/decoration/random steps | erste Graphik
  shadows%
}

\usepackage{listings}
\usepackage{verbatim}


%\usepackage{pscyr}						% Красивые русские шрифты

%%% Математические пакеты %%%
\usepackage{amsfonts,amssymb,amscd} % Математические дополнения от AMS

%%% Оформление абзацев %%%
\usepackage{indentfirst} % Красная строка

%%% Цвета %%%
%\usepackage[usenames]{color}
%\usepackage{color}

%%% Таблицы %%%
\usepackage{tabularx}
\usepackage{longtable}					% Длинные таблицы
\usepackage{multirow,makecell,array}	% Улучшенное форматирование таблиц

%%% Общее форматирование
\usepackage[singlelinecheck=off,center]{caption}	% Многострочные подписи
\usepackage{soul}									% Поддержка переносоустойчивых подчёркиваний и зачёркиваний

%%% Библиография %%%
%\usepackage{cite} % Красивые ссылки на литературу

%%% Гиперссылки %%%
%\usepackage[linktocpage=true,plainpages=false,pdfpagelabels=false]{hyperref}

%%% Изображения %%%
\usepackage{graphicx} % Подключаем пакет работы с графикой

%%% Оглавление %%%
%\usepackage[subfigure]{tocloft}
\usepackage{pdftexcmds}
%\usepackage[svgpath=fig/]{svg}
\usepackage{float}
\usepackage{wrapfig}
\usepackage{sidecap}

\usepackage{chemfig}


\usepackage{arydshln}

%%%% Notes
\usepackage{pgfpages}
%\setbeameroption{show notes on second screen}
%%%% EndNotes
%\usepackage{subfigure}

\newcolumntype{P}[1]{>{\raggedright\arraybackslash}p{#1}}
\newcolumntype{Z}[1]{>{\raggedleft\arraybackslash}p{#1}}
\newcolumntype{Y}[1]{>{\centering\arraybackslash}p{#1}}
\newcommand{\tss}[1]{\textsuperscript{#1}}
\newcommand{\tbs}[1]{\textsubscript{#1}}
\newcommand{\dg}{\ensuremath{^\circ}}

\newcommand{\paral}[2]{%
    \begin{tikzpicture}[scale=#2]%
          \draw [ color=black, thick, fill=#1 ] (0,0) %
    --(1,0)--(1.5,1)--(0.5,1)-- (0,0);%
          \end{tikzpicture}
            }
\newcommand{\romb}[2]{%
    \begin{tikzpicture}[scale=#2]%
          \draw [ color=black, thick, fill=#1 ] (0,0) %
    --(1,0)--(1.5,0.866)--(1,1.732)--(0,1.732)--(-0.5,0.866)  -- (0,0);%
          \end{tikzpicture}
            }
\newcommand{\wat}[1]{%
\begin{tikzpicture}[xscale=#1,yscale=#1]
        \draw [ color   = black,   thick,   ] (1,2) -- (2,1);
        \draw [ color   = black,   thick,   ] (3,2) -- (2,1);
        \draw[fill=white] (1,2) circle [radius=.3];  
        \draw[fill=white] (3,2) circle [radius=.3];  
        \draw[fill=red!80] (2,1) circle [radius=.3];  
\end{tikzpicture}
}
\newcommand{\disk}[1]{%
   \begin{tikzpicture}\draw[fill=#1,overlay,#1](0,0)circle(2pt);
   \end{tikzpicture} 
}
\newcommand{\ddisk}[2]{%
\begin{tikzpicture}\draw[fill=#1,overlay,#1] (0,#2pt) circle (#2pt);
\end{tikzpicture} 
}

\newcommand{\gua}[2]{%
%         \setatomsep{2.0em}\setcrambond{0.2em}{}{}%
         \chemfig[atom sep=1.5em,cram width=0.2em]{%
         \ddisk{magenta}{4}-[:#2]\disk{blue}*5(-=@{n7#1}\disk{blue}-(*6(-(=@{o#1}\disk{red})-\disk{blue}(-@{n#1}\disk{gray})-(%
         -\disk{blue}(-[::60]\disk{gray})(-[::-60]@{n2#1}\disk{gray}))=\disk{blue}-))=-)%     
               }}
\newcommand{\includegraphicsfs}[1]{%
        \begin{tikzpicture}[remember picture,overlay]
                \node[at=(current page.center),fill=white] {
                 \includegraphics[width=\paperwidth]{#1}
                   };
        \end{tikzpicture}
  }

%\usepackage{luacode}
%\begin{luacode*}
%function printtable(k)
%     n=string.len(k)
%     for i=1,n do 
%      for j=1,n do 
%        if (string.sub(k,i,i) == "G" and string.sub(k,j,j) == "C") or 
%        (string.sub(k,i,i) == "C" and string.sub(k,j,j) == "G") then  
%        tex.sprint("\\node at (".. i/5 ..",".. j/5 ..") {".. 3 .."};" )
%        end
%        if (string.sub(k,i,i) == "A" and string.sub(k,j,j) == "U") or 
%        (string.sub(k,i,i) == "U" and string.sub(k,j,j) == "A") then  
%        tex.sprint("\\node at (".. i/5 ..",".. j/5 ..") {"..  2 .."};" )
%        end
%       end
%     end 
%     tex.print("\\draw[blue,thick] (0.2,".. n/5 ..") -- (".. n/5+0.2 ..",0);")
%end
%\end{luacode*}
%\begin{luacode*}
%function printtableold(k)
%%     n=string.len(k)
%     for i=1,n do 
%      for j=1,n do 
%        if (string.sub(k,i,i) == "G" and string.sub(k,j,j) == "C") or 
%        (string.sub(k,i,i) == "C" and string.sub(k,j,j) == "G") then  
%        tex.sprint("\\node at (".. i/5 ..",".. j/5 ..") {".. string.sub(k,i,i).."};" )
%        end
%        if (string.sub(k,i,i) == "A" and string.sub(k,j,j) == "U") or 
%        (string.sub(k,i,i) == "U" and string.sub(k,j,j) == "A") then  
%        tex.sprint("\\node at (".. i/5 ..",".. j/5 ..") {".. string.sub(k,i,i).."};" )
%        end
%       end
%     end 
%     tex.print("\\draw[blue,thick] (0.2,".. n/5 ..") -- (".. n/5+0.2 ..",0);")
%end
%\end{luacode*}
%
%
%
%
%
%

\usepackage{../sty/header}
\usepackage{../sty/pdfpc-commands}
\graphicspath{{../fig/}{../img/}{../../hse/img/}{../../hse/l1/fig/}}

\title []%[Термостаты и Баростаты] % (optional, use only with long paper titles)
{Лекция 1. Введение в структуру белка, Молекулярная механика и квантовая химия }
\subtitle{Курс: Машинное обучение в структурной биологии}
\author[Головин А.В.]{Головин А.В. \inst{1}}

\institute[МГУ]% (optional, but mostly needed)
{\inst{1}%
  МГУ им М.В. Ломоносова, Факультет Биоинженерии и Биоинформатики 
   }

% - Use the \inst command only if there are several affiliations.
% - Keep it simple, no one is interested in your street address.

\date[Осень, \the\year] % (optional, should be abbreviation of conference name)
{Москва, \the\year}
\subject{Моделирование структуры белков}
% This is only inserted into the PDF information catalog. Can be left
% out. 


 \pgfdeclareimage[height=0.5cm]{university-logo}{../msu-logo.png}
 \logo{\pgfuseimage{university-logo}}

% Delete this, if you do not want the table of currents to pop up at
% the beginning of each subsection:
%\AtBeginSubsection[]
%{\begin{frame}<beamer>{Содержание}
%    \tableofcurrents[currentsection,currentsubsection]
%  \end{frame}
%}


% If you wish to uncover everything in a step-wise fashion, uncomment
% the following command: 

%\beamerdefaultoverlayspecification{<+->}


\begin{document}

\begin{frame}[plain]
  \titlepage
\end{frame}




\section{Введение}

\begin{frame}
    {Структура рецептора}
	\begin{center}
          \includegraphics[width=0.85\textwidth]{prot.png}
	  \end{center}
  \end{frame}


\begin{frame}
    {Что такое белок?}{}
	\textbf{Белки}  — высокомолекулярные органические вещества, состоящие из соединённых в цепочку пептидной связью альфа-аминокислот.(wikipedia) \\
	\vspace{.5cm}
	\begin{center}
	\small%
	\chemfig[atom sep=2em]{-N(-[2]H)-C(-[2]H)(-[6]R_1)-C(=[2]O)-N(-[6]H)-C(-[6]H)(-[2]R_2)-C(=[6]O)-N(-[2]H)-C(-[2]H)(-[6]R_3)-C(=[2]O)-}\\
	\vspace{.5cm}
    \end{center}
	\textbf{Или:} белок это линейный полярный полимер, где мономерами является выборка из примерно 20 L-альфа-аминокислот. 

\end{frame}

\setchemfig{atom sep=2em, cram width=0.2em}

\begin{frame}
    {Что такое  L альфа-аминокислота?}{}
	\begin{center}
%\setatomsep{2.7em}\setcrambond{0.2em}{}{}%
 \chemfig{C(-[5]H)(-[2]H)(<[:-60]H)(<:[:-20]H)} \\
 \small
 атом углерода в $sp^3$ гибридизации имеет тетраэдрическое окружение\\
  \vspace{.5cm}
 \chemfig{C(-[5]NH)(-[7]CO)(<[:60]R_1)(<:[:120]H)}
 \hspace{1cm}
 \chemfig{C(-[5]NH)(-[7]CO)(<:[:60]R_1)(<[:120]H)}\\
 \vspace{.2cm}
 L-аминокислота \hspace{1cm} D-аминокислота \\
 \end{center}
% \chemfig{C(-[5]NH)(-[2]R_1)(<[:-60]CO)(<:[:-20]H)}
\end{frame}

\begin{frame}
    {Аминокислоты}
	\begin{center}
          \includegraphics[width=.9\textwidth]{Amino_Acids}
          %[height=.7\textheight]{Amino_Acids}
    \end{center}
  \end{frame}


\begin{frame}
    {Пептидная связь}
	\begin{center}
%		         \setlength{\fboxsep}{1pt}
%				            \fcolorbox{black}{white}{ 
%          \includegraphics[width=.8\textwidth]{pep1.eps}
  \begin{tikzpicture}[help lines/.style={thin,draw=black!50}]
%  \draw[help lines] (0,0) grid (8,4);    
   \node (2) at (4,2) {
   % \schemedebug{true}
     %\setatomsep{2.7em}\setcrambond{0.2em}{}{}%
     %\setbondstyle{line width=1pt}
     \chemfig{
         H_2N-[1](-[2]R_1)-[7]C(=[6]O)-[1,,,,white!40!blue,{line width=2pt}]N(-[2]H)-[7](-[6]R_2)-[1]C(=[2]O)-[7,,,,white!40!blue,{line width=2pt}]N(-[6]H)
         -[1](-[2]R_3)-[7]C(=[6]O)-[1]OH
         }
	  };
      \draw[orange] (1.5,1.8) ellipse (1.3cm and 2cm);
      \draw[orange] (3.7,2.3) ellipse (1.2cm and 2cm);
      \draw[orange] (5.9,1.8) ellipse (1.2cm and 2cm);
      \node (pep) at (4,-1) {Пептидные связи};
      \draw [->,white!40!blue,thick] (pep.north) -- (2.8,2);
      \draw [->,white!40!blue,thick] (pep.north) -- (4.8,2);
      %node[ellipse, minimum height=4cm,minimum width=2cm,draw] {};
  \end{tikzpicture}
 \end{center}
\end{frame}

\begin{frame}
    {Пептидная связь, таутомерия}
	\begin{center}
	 \schemestart
	 \small%
%		 \setatomsep{2em}%
%\setatomsep{2.7em}\setcrambond{0.2em}{}{}%
	 \chemfig{C(-[5])(<:[:60]H)(<[:120]R_1)-[7]C(=[6]O)-[1,,,,white!40!blue,{line width=1pt}]N(-[2]H)-[7]C(<:[:-60]R_2)(<[:-120]H)-[1] }%
	 \arrow{<=>}%
%\setatomsep{2.7em}\setcrambond{0.2em}{}{}%
	 \small\chemfig{C(-[5])(<:[:60]H)(<[:120]R_1)-[7]C(-[6]OH)=[1,,,,white!40!red,{line width=1pt}]N-[7]C(<:[:-60]R_2)(<[:-120]H)-[1] }%
	  \schemestop
    \end{center}
\end{frame}

\begin{frame}
{Пептидная связь, свойства}
	\begin{itemize}
		\item Пептидная связь прочнее, чем другие амиды
		\item Атомы пептидного звена ( C$_\alpha$-C-N- C$_\alpha$) лежат в одной плоскости
    \item Валентные углы у атомов С и N примерно равны $120^o$
		\item Вращение вокруг связи C-N затруднено
		\item Возможны cis- и trans-конфигурации; в белках преобладают trans
		\item Карбонильный кислород – хороший акцептор водорода
		\item Амидный азот – хороший донор водорода
		\end{itemize}
 \end{frame}

 \begin{frame}
{Вращения вокруг связей в остове белка}
	\begin{center}
%`\setatomsep{3.7em}\setcrambond{0.2em}{}{}%
	 \chemfig{C(-[5])(<:[:60]H)(<[:120]R_1)-[7]C(=[6]O)-[@{om}1]N
     (-[2]H)-[@{phi}7]C(<:[:-60]R_2)(<[:-120]H)-[@{psi}1]C(=[2]O)-[7]}%
 \chemmove{%
     \draw[-stealth,thick,white!40!red]
     (phi).. controls +(45:5mm) and +(-90:5mm).. node[above right]
     {$\boldsymbol{\phi}$}(phi);
     \draw[-stealth,thick,white!40!blue]
     (psi).. controls +(135:5mm) and +(-90:5mm).. node[below right]
     {$\boldsymbol{\psi}$}(psi);
     \draw[-stealth,thick,green!60!black]
     (om).. controls +(135:5mm) and +(-90:5mm).. node[above left]
     {$\boldsymbol{\omega}$}(om);
 }
	  \end{center}
  \end{frame}




  \begin{frame}
{Вращения вокруг связей в остове белка}
	\begin{center}
%\setatomsep{3.7em}\setcrambond{0.2em}{}{}%
	 \chemfig{C(-[5])(<:[:60]H)(<[:120]R_1)-[7]C(=[6]O)-[@{om}1]N
     (-[2]H)-[@{phi}7]C(<:[:-60]R_2)(<[:-120]H)-[@{psi}1]C(=[2]O)-[7]}%
 \chemmove{%
     \draw[-stealth,thick,white!40!red]
     (phi).. controls +(45:5mm) and +(-90:5mm).. node[above right]
     {$\boldsymbol{\phi}$}(phi);
     \draw[-stealth,thick,white!40!blue]
     (psi).. controls +(135:5mm) and +(-90:5mm).. node[below right]
     {$\boldsymbol{\psi}$}(psi);
     \draw[-stealth,thick,green!60!black]
     (om).. controls +(135:5mm) and +(-90:5mm).. node[above left]
     {$\boldsymbol{\omega}$}(om);
 }
	  \end{center}

 $ \begin{array}{l}  \phi\\ \psi  \end{array}  \Bigg \} $ теоретически могут быть: от –180$^0$ до +180$^0$ \\
 \vspace{.5cm}
	  а $\omega$ ?
  \end{frame}

  \begin{frame}
{Карта Рамачандрана}
	  даже в полиглициновой цепи существуют стерические ограничения
	\begin{center}
          \includegraphics[width=0.75\textwidth]{ram_gly.png}
	  \end{center}
  \end{frame}
 
  \section{Уровни организации структуры белка}
  \begin{frame}
{Уровни организации структуры белка}
	  \begin{itemize}
		  \item Первичная структура
		  \item Вторичная структура
		  \item Укладка (fold)
		  \item Третичная структура
		  \item Четвертичная структура
		  \end{itemize}
	  \end{frame}

  \begin{frame}
{Первичная структура}
	  Первичная структура – это аминокислотная последовательность:\\
	  \vspace{.5cm}
	  Met-Ala-Gly-Trp-Ala-Val-Asp \ldots
  \end{frame}

  \begin{frame}
{Вторичная структура}
	  \begin{wrapfigure}{r}{2cm}
	  \includegraphics[width=2cm]{ab.png}\\
	  \includegraphics[width=2cm]{bturn.png}
      \end{wrapfigure}
	  \textbf{Вторичная структура белка} - это упорядоченные расположения атомов основной цепи полипептида, 	 безотносительно к типам боковых цепей (групп) и их конформациям.\\
	  \vspace{.5cm}
	  Если упорядоченность такова, что двугранные углы
	  одинаковы у всех остатков, то говорят о
	  регулярной вторичной структуре. Регулярными
	  вторичными структурами являются спирали и $\beta$–
	  структуры.\\
	  \vspace{.7cm}
	  Пример нерегулярной вторичной структуры 
	  $\beta$–поворот ($\beta$–изгиб, реверсивный поворот).
  \end{frame}


  \begin{frame}
{Регулярные вторичные структуры}
	\begin{center}
%		         \setlength{\fboxsep}{3pt}
%				            \fcolorbox{black}{white}{ 
          \includegraphics[width=0.8\textwidth]{ss.png}
%	  }
	  \end{center}
	\end{frame}

	\begin{frame}
{Укладка (fold)}
		Укладкой называют организацию в пространстве элементов регулярной вторичной структуры. \\
		Пример: $\alpha$ -спиральные белки\\
	\begin{center}
          \includegraphics[width=0.2\textwidth]{ss1.png}
          \includegraphics[width=0.2\textwidth]{ss2.png}
          \includegraphics[width=0.2\textwidth]{ss3.png}
	  \end{center}
	\end{frame}

\begin{frame}
{\texorpdfstring{$\beta$} - структурные белки}
	\begin{center}
		         %\setlength{\fboxsep}{0pt}
				  %          \fcolorbox{black}{white}{ 
								\begin{minipage}{\textwidth}
         % \includegraphics[width=0.5\textwidth]{b1.png}
         % \includegraphics[width=0.5\textwidth]{b2.png} \\
          \includegraphics[height=0.6\textheight]{b3.png}
          \includegraphics[height=0.6\textheight]{b4.png}
	  \end{minipage}
	  %}
	  \end{center}
	\end{frame}

	\begin{frame}
{Распределение в природе}
	\begin{center}
          \includegraphics[width=0.4\textwidth]{scheme1.png}
	  \end{center}
	\end{frame}

	\begin{frame}
{Третичная структура}
		Третичной структурой называют расположение в пространстве всех атомов
		одной полипептидной цепи. \\
		Т.e. описание третичной структуры включает в себя:
		\begin{itemize}
			\item описание элементов вторичной структуры,
			\item описание типа укладки,
			\item описание структуры петель,
			\item описание конформаций боковых групп всех аминокислотных остатков.
			\end{itemize}
		\end{frame}
\section{Типы взаимодействий в белках}

\begin{frame}
{Вспомогательные взаимодействия: водородные связи}
	\begin{center}
        \chemfig{
            -[5]N-[7]C_\alpha(-[5]C(-[7])=[4]O)--[1]@{o1}O-H
            -[7,,,,dashed]@{o2}\Charge{120=\:,-120=\:}{O}=C(-[::60]\Charge{60=\-}{O})-[::-60]--[::-60]
        }
  \chemmove[dashed,white!40!red]{
  \draw[->] (o1) -- (o2)  node[below left] {3.5\AA\hspace{1cm}~};}

		%         \setlength{\fboxsep}{3pt}
	%			            \fcolorbox{black}{white}{ 
    %      \includegraphics[width=0.7\textwidth]{hbond.eps}
	%  }
	  \end{center}
  \end{frame}

\begin{frame}
{Ионные пары}
	\begin{center}
        \chemfig{
            --[1]--[7]@{o1}\chemabove{NH_3}{\oplus}
            -[:-90,1.5,,,draw=none,white!40!red]@{o3}\Charge{120=\:,-120=\:}{O}=[::90]C(-[::60]@{o2}O^{-})
            -[::-60]-[::60]-[::-60]
        }
  \chemmove[dashed,white!40!red]{
  \draw[->] (o1) -- (o2)  node[below left] {~};
  \draw[->] (o1) -- (o3)  node[below left] {3.1\AA\hspace{1cm}~};}
	  \end{center}
  \end{frame}

\begin{frame}
{Дисульфидные мостики характерны для секретируемых белков}
	\begin{center}
        \chemfig{
            -[5]N-[7]C_\alpha(-[5]C(-[7])=[4]O)--[1]S-[0,,,,yellow!60!black,
            thick]S-[7]
            -[0]C_\alpha(-[1])-[7]
        }
	  \end{center}
  \end{frame}

\begin{frame}
{Гидрофобные взаимодействия – главный 	фактор, заставляющий глобулу свертываться}
	\begin{center}
  \begin{tikzpicture}[help lines/.style={thin,draw=black!50}]
  %\draw[help lines] (0,0) grid (8,4);    
   \node (2) at (4,2) {
     \chemfig{
         (-[1,0.5,,,draw=none]-[1,2,,,decorate,decoration=snake]COO^{-})
         (-[2,0.5,,,draw=none]-[2,2,,,decorate,decoration=snake]COO^{-})
         (-[0,0.5,,,draw=none]-[0,2,,,decorate,decoration=snake]COO^{-})
         (-[7,0.5,,,draw=none]-[7,2,,,decorate,decoration=snake]COO^{-})
         (-[5,0.5,,,draw=none]-[5,2,,,decorate,decoration=snake]^{-}OOC)
         (-[3,0.5,,,draw=none]-[3,2,,,decorate,decoration=snake]^{-}OOC)
         (-[4,0.5,,,draw=none]-[4,2,,,decorate,decoration=snake]^{-}OOC)
         (-[6,0.5,,,draw=none]-[6,2,,,decorate,decoration=snake]COO^{-})
 }};
      \draw[orange] (4,2) ellipse (2cm and 2cm);
      \node (pep) at (5,0) {$\approx$ 4\AA};
 \end{tikzpicture}

	  \end{center}
  \end{frame}




\begin{frame}
{От четвертичной структуры к молекулярным машинам}
	\begin{center}
%		         \setlength{\fboxsep}{0pt}
%				            \fcolorbox{black}{white}{ 
          \includegraphics[height=0.6\textheight]{4s1.png}
          \includegraphics[height=0.6\textheight]{4s2.png}
%	  }
	  \end{center}
  \end{frame}
	



\begin{frame}
{Вспомогательные взаимодействия: водородные связи}
	\begin{center}
        \setlewis[]{}{}{fill=red, draw=red}
        \chemfig{
            -[5]N-[7]C_\alpha(-[5]C(-[7])=[4]O)--[1]@{o1}O-H
            -[7,,,,dashed]@{o2}\textcolor{white}{\lewis{3:5:,O}}=C(-[::60]O^{-})
            -[::-60]--[::-60]
        }
  \chemmove[dashed,white!40!red]{
  \draw[->] (o1) -- (o2)  node[below left] {3.5\AA\hspace{1cm}~};}

		%         \setlength{\fboxsep}{3pt}
	%			            \fcolorbox{black}{white}{ 
    %      \includegraphics[width=0.7\textwidth]{hbond.eps}
	%  }
	  \end{center}
  \end{frame}


\end{document}


